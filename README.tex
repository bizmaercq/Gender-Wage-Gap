\documentclass[]{article}
\usepackage{lmodern}
\usepackage{amssymb,amsmath}
\usepackage{ifxetex,ifluatex}
\usepackage{fixltx2e} % provides \textsubscript
\ifnum 0\ifxetex 1\fi\ifluatex 1\fi=0 % if pdftex
  \usepackage[T1]{fontenc}
  \usepackage[utf8]{inputenc}
\else % if luatex or xelatex
  \ifxetex
    \usepackage{mathspec}
  \else
    \usepackage{fontspec}
  \fi
  \defaultfontfeatures{Ligatures=TeX,Scale=MatchLowercase}
\fi
% use upquote if available, for straight quotes in verbatim environments
\IfFileExists{upquote.sty}{\usepackage{upquote}}{}
% use microtype if available
\IfFileExists{microtype.sty}{%
\usepackage{microtype}
\UseMicrotypeSet[protrusion]{basicmath} % disable protrusion for tt fonts
}{}
\usepackage[margin=1in]{geometry}
\usepackage{hyperref}
\hypersetup{unicode=true,
            pdftitle={Gender Wage Gap},
            pdfauthor={Bismark Adomako},
            pdfborder={0 0 0},
            breaklinks=true}
\urlstyle{same}  % don't use monospace font for urls
\usepackage{graphicx,grffile}
\makeatletter
\def\maxwidth{\ifdim\Gin@nat@width>\linewidth\linewidth\else\Gin@nat@width\fi}
\def\maxheight{\ifdim\Gin@nat@height>\textheight\textheight\else\Gin@nat@height\fi}
\makeatother
% Scale images if necessary, so that they will not overflow the page
% margins by default, and it is still possible to overwrite the defaults
% using explicit options in \includegraphics[width, height, ...]{}
\setkeys{Gin}{width=\maxwidth,height=\maxheight,keepaspectratio}
\IfFileExists{parskip.sty}{%
\usepackage{parskip}
}{% else
\setlength{\parindent}{0pt}
\setlength{\parskip}{6pt plus 2pt minus 1pt}
}
\setlength{\emergencystretch}{3em}  % prevent overfull lines
\providecommand{\tightlist}{%
  \setlength{\itemsep}{0pt}\setlength{\parskip}{0pt}}
\setcounter{secnumdepth}{0}
% Redefines (sub)paragraphs to behave more like sections
\ifx\paragraph\undefined\else
\let\oldparagraph\paragraph
\renewcommand{\paragraph}[1]{\oldparagraph{#1}\mbox{}}
\fi
\ifx\subparagraph\undefined\else
\let\oldsubparagraph\subparagraph
\renewcommand{\subparagraph}[1]{\oldsubparagraph{#1}\mbox{}}
\fi

%%% Use protect on footnotes to avoid problems with footnotes in titles
\let\rmarkdownfootnote\footnote%
\def\footnote{\protect\rmarkdownfootnote}

%%% Change title format to be more compact
\usepackage{titling}

% Create subtitle command for use in maketitle
\providecommand{\subtitle}[1]{
  \posttitle{
    \begin{center}\large#1\end{center}
    }
}

\setlength{\droptitle}{-2em}

  \title{Gender Wage Gap}
    \pretitle{\vspace{\droptitle}\centering\huge}
  \posttitle{\par}
    \author{Bismark Adomako}
    \preauthor{\centering\large\emph}
  \postauthor{\par}
      \predate{\centering\large\emph}
  \postdate{\par}
    \date{April 17, 2020}


\begin{document}
\maketitle

\hypertarget{goals}{%
\subsection{Goals}\label{goals}}

Here we ask and answer the following question: What is the difference in
predicted wages between men and women with the same job-relevant
characteristics?

\hypertarget{content-of-repo}{%
\subsection{Content of Repo}\label{content-of-repo}}

\begin{enumerate}
\def\labelenumi{\arabic{enumi}.}
\tightlist
\item
  data/codebook.rtf contains the description of worker job-relevant
  worker characteristics.
\item
  data/pay.descrimination.Rdata: the CPS (2012) data on wages and
  job-relevant worker characteristics, suc as experience exp, gender,
  education, etc.
\item
  gender\_wage\_gap.R predicts the difference in predicted wages between
  men and women with the same job-relevant characteristics using linear
  model with linear specificaions.
\end{enumerate}

\hypertarget{data-summary}{%
\subsection{Data Summary}\label{data-summary}}

\begin{verbatim}
       male averages female averages
female     0.0000000       1.0000000
cg         0.3548387       0.4061135
sc         0.3019713       0.3543356
hsg        0.3431900       0.2395508
mw         0.2849462       0.2913288
so         0.2352151       0.2551466
we         0.2213262       0.1983780
ne         0.2585125       0.2551466
exp1      13.5801971      13.0371179
exp2       2.5865883       2.4494526
exp3       5.9649377       5.5992975
wage      16.1174580      14.7200584
\end{verbatim}

\hypertarget{specifications}{%
\subsection{Specifications}\label{specifications}}

We estimate the linear regression model:

\begin{verbatim}
    Y = β1D + βr W + ε
    
\end{verbatim}

D is the indicator of being a female (1 if female and 0 otherwise). W's
are controls. Basic model: W's consist of education and regional
indicators, experience, experience squared, and experience cubed.
Flexible model: W's consist of controls in the basic model plus all of
their two-way interactions.

\hypertarget{results}{%
\subsection{Results}\label{results}}

\begin{verbatim}
           Estimate Standard Error Lower Conf. Bound Upper Conf. Bound
basic reg -1.826397      0.4245163         -2.658697        -0.9940968
flex reg  -1.880013      0.4247438         -2.712761        -1.0472652
\end{verbatim}

The estimated gender gap in hourly wage is about \$−2.0 with a
confidence interval that ranges from about \$−2.7 to \$−1. This means
that women get paid \$2 less per hour on average than men, controlling
for experience, education, and geographicalregion.

\hypertarget{conclusion}{%
\subsection{Conclusion}\label{conclusion}}

To sum it up\ldots{} The gender wage gap may partly reflect genuine
discrimination against women in the labormarket\ldots{}. It may also
partly reflect the so-called selection effect, namely that women are
more likely to end up in occupations that pay somewhat less (for
example, school teachers).


\end{document}
